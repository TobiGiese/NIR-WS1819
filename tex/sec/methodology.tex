% !TeX spellcheck = de_DE
\section{Methodik}
\label{sec:Methodik}
	
	\subsection{Datensatz}
	\label{ssec:Datensatz}
	
	    Der für die Modellwahl verwendete Datensatz beinhaltet die logarithmierten relativen Reflexionswerte $-\delta(\lambda)$ bei Wellenlängen zwischen 1400 nm und 2672 nm in einem Abstand von 4 nm sowie die Stoffmengeanteile $y^{(N)}$, $y^{(SOC)}$ und entsprechenden pH-Werte von insgesamt 533 Proben.
	    Informationen über die chemische Zusammensetzung der Probe in den Reflexionswerten im Nahinfrarotbereich sind stark überlagert. \cite{Agelet2010} 
	    Aus diesem Grund ist eine sorgfältige Auswahl relevanter Wellenlängen von besonderer Bedeutung für die Erstellung eines zuverlässigen Modells. 
	    
	
	% subsection Datensatz

	\subsection{Statistisches Modell}
	\label{ssec:Statistisches Modell}
	
	    Sei $n\in\SN$ die Größe des Datensatzes und $k\in\SN$ mit $k< n$ die Anzahl der Wellenlängen im Datensatz. 
	    Entsprechend Abschnitt \ref{ssec:Datensatz} definieren wir dann die Einflussgröße $x_{ij}$ als für die $i$te Probe und $j$te Wellenlänge als
	    \[
			x_{ij} \define -\lg \delta_i(\lambda_j)
		\]
		für jedes $i,j\in\SN,i\leq n,j\leq k$.
	    
	    Der Stoffmengenanteil des Stickstoffs  $y^{(N)}$ stellt die Zielgröße unseres späteren Modells dar.
	    Wie definieren hierfür den Vektor $y_i$ den Stoffmengenanteil der $i$ten Probe als den $n$ dimensionalen Vektor
		\[
			 y^{(N)} \define \curvb{y^\m{(N)}_i}
		\]

        Nachdem wir sowohl die Einflussgrößen als auch die Zielgröße für das lineare Modell definiert haben lassen sich diese nun in Zusammenhang bringen.
        Es ist plausibel anzunehmen, die sich sie Zielgröße durch einen Linearkombination der Einflussgrößen beschreiben lässt.
        Hierfür definieren wir zunächst $Y^\m{(N)}$ als einen zufälligen Vektor von $y^\m{(N)}$
        \[ 
			 \expect Y^\m{(N)} \define \beta_0 + \sum_{j=1}^k{x_{ij}\beta_j} 
		\]
		
		Zudem ist es notwendig eine Variable $\varepsilon^\m{(N)}$ einzuführen welchen den Zufall der Messungen beschreibt.
	    In Matrixschreibweise lässt sich dies als die Designmatrix $\mathbb{X} \in \SR^{n \times (k+1)}$, dem Parametervektor $\beta \in \SR^{k+1}$ und dem stochastisch  verteilten Parameter $\varepsilon^\m{(N)}$ darstellen, sodass gilt,
		\[
			Y^\m{(N)} = \mathbb{X}\beta + \varepsilon^\m{(N)}
		\]
		mit 
		\[
			\expect \varepsilon^\m{(N)} = 0, \qquad \cov \varepsilon^\m{(N)} = (\sigma^2)^\m{(N)} \idmat
		\]
		wobei $(\sigma^2)^\m{(N)} \in (0,\infty)$.
		Weiterhin soll angenommen werden, dass $\varepsilon^\m{(N)}$ normalverteilt ist mit
	    \[
			\varepsilon^\m{(N)} \sim \FN \curvb{0,(\sigma^2)^\m{(N)}\idmat}
		\]
	    sodass sich für das Gesamtmodell gilt 
		\[
			Y^\m{(N)} \sim \FN \curvb{\mathbb{X}\beta^\m{(N)},(\sigma^2)^\m{(N)} \idmat}
		\]
	
	% subsection Statistisches Modell

	\subsection{Modellwahl im klassischen linearen Modell}
	\label{ssec:mlr}
	Im klassischen linearen Modell wird meist ... (s. Skript, S. XY)
	Das Problem mit der Wahl von $\alpha$ ...
	Beim Hinzufügen neuer Einflussgrößen in das Modell wird die Nullhypothese $H_0$ ...
	
		

	% subsection mlr

	\subsection{Mallow's $C_{p}$}
	\label{ssec:mallows-C_p}
	
		At this point, the model is specified using $k+1 = 320$ predictors for each response variable, using the whole domain of measured spectrum for each soil sample.
		
		
	
	
	% subsection mallows-C_p

		
	\subsection{Model Validation}
	\label{ssec:model-validation}
	
	
	% subsection model-validation
	
	\subsection{Assessment by Simulation}
	\label{ssec:simulation}
	
		
		
	% subsection simulation
% section methodology