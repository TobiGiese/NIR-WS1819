
	
		\begin{figure*}
			\center
			% \subcaphangtrue
			\begin{subfigure}[t]{0.33\textwidth}
				\centerline{
					\input{gp/ms-sa-soc-corr}
				}
				\caption{\parbox[t]{0.45\textwidth}{
					$\hat{p}^{(\m{SOC})} \sim p^{(\m{SOC})}$ \smallskip \\
					$(R^2)^{(\m{SOC})} = 0.871$
					}}
			\end{subfigure}
			\begin{subfigure}[t]{0.33\textwidth}
				\centerline{
					\input{gp/ms-sa-n-corr}
				}
				\caption{\parbox[t]{0.4\textwidth}{
					$\hat{p}^{(\m{N})} \sim p^{(\m{N})}$ \smallskip \\
					$(R^2)^{(\m{N})} = 0.861$
					}}
			\end{subfigure}
			\begin{subfigure}[t]{0.33\textwidth}
				\centerline{
					\input{gp/ms-sa-ph-corr}
				}
				\caption{\parbox[t]{0.43\textwidth}{
					$\widehat{\m{pH}} \sim \m{pH}$ \smallskip \\
					$(R^2)^{(\m{pH})} = 0.940$
					}}
				\label{sfig:gof-ph}
			\end{subfigure}
			\caption{Correlation diagrams plotting $\hat{y}$ on $y$ and the blue line representing the $\id$}
			\label{fig:gof}
		\end{figure*}

\section{Calibration}
\label{sec:model-selection}
	
	\subsection{Model Selection}
	\label{ssec:calibration}
		
	
				
	% subsection calibration

	\subsection{Bewertung der Modellwahl}
	\label{ssec:Bewertung der Modellwahl}
	    Anhand des $R^2 = 0.82$ zeigt sich, dass das gewählte Modell die Varianz in den Daten gut erklären kann.
	    Dies spiegelt auch das Korrelationsdiagramm wieder, in welchen der wahre Wert auf der x-Achse und der geschätzte Stickstoffmengenanteil auf der y-Achse aufgetragen sind.
	    Abweichungen sind lediglich im unteren Bereich zwischen $0.0$ und $0.03$ und ab $0.3$ zu beobachten, wo der Stoffmengenanteil des Stickstoffs durch das Modell unterschätzt wird.
	    Dies lässt dich vermutlich durch die geringe Datendichte in diesen Bereichen erklären.
	    Insgesamt kann man sagen, dass die bisherigen Untersuchungen auf ausreichend gute Prognosen durch das Modell hindeuten.
		
		
	% subsection Bewertung der Modellwahl

% section model-selection