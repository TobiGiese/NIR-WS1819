
	
		\begin{figure*}
			\center
			% \subcaphangtrue
			\begin{subfigure}[t]{0.33\textwidth}
				\centerline{
					\input{gp/ms-sa-soc-corr}
				}
				\caption{\parbox[t]{0.45\textwidth}{
					$\hat{p}^{(\m{SOC})} \sim p^{(\m{SOC})}$ \smallskip \\
					$(R^2)^{(\m{SOC})} = 0.871$
					}}
			\end{subfigure}
			\begin{subfigure}[t]{0.33\textwidth}
				\centerline{
					\input{gp/ms-sa-n-corr}
				}
				\caption{\parbox[t]{0.4\textwidth}{
					$\hat{p}^{(\m{N})} \sim p^{(\m{N})}$ \smallskip \\
					$(R^2)^{(\m{N})} = 0.861$
					}}
			\end{subfigure}
			\begin{subfigure}[t]{0.33\textwidth}
				\centerline{
					\input{gp/ms-sa-ph-corr}
				}
				\caption{\parbox[t]{0.43\textwidth}{
					$\widehat{\m{pH}} \sim \m{pH}$ \smallskip \\
					$(R^2)^{(\m{pH})} = 0.940$
					}}
				\label{sfig:gof-ph}
			\end{subfigure}
			\caption{Correlation diagrams plotting $\hat{y}$ on $y$ and the blue line representing the $\id$}
			\label{fig:gof}
		\end{figure*}

\section{Calibration}
\label{sec:model-selection}
	
	\subsection{Model Selection}
	\label{ssec:calibration}
		
	
				
	% subsection calibration

	\subsection{Bewertung der Modellwahl}
	\label{ssec:Bewertung der Modellwahl}
	    Anhand des $R^2$ Werte zeigt sich, dass das gewählte Modell ... die Varianz in den Daten gut erklären kann.
	    Dies spiegelt auch das Korrelationsdiagramm wieder bei dem....
	    Abweichungen sind jedoch im dem Bereich .... zu beobachten, was durch ... erklärt werden kann.
	    Insbesondere werden die Stickstoffwerte im dem Bereich zwischen... unter / überschätzt.
	    
	    Insgesamt kann man sagen, dass das gewählte Modell ausreichend gute Prognosen liefert....
		
		
	% subsection Bewertung der Modellwahl

% section model-selection