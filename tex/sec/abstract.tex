In der vorliegenden Arbeit wird eine Methode zur Ableitung einer Kalibierfunktion zur Bestimmung des Stickstoffgehalts in Bodenproben vorgestellt.
Basierend auf den zugrunde liegenden physikalischen und chemischen Eigenschaften der Nahinfrarotspektroskopie wird ein lineares Modell formuliert. 
Hierfür wurden potenzielle Prediktoren entsprechend ihrer Variabilität für das Maximalmodell ausgewählt und dann mit Hilfe von Mallow's Cp ein Idealmodell ermittelt. 
Darüber hinaus wurde der Einfluss des Stichprobenumfangs auf den durch den minimalen Cp-Wert geschätzten Vorhersagefehler untersucht.