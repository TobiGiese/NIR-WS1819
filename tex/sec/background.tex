\section{Hintergrund}
\label{sec:Hintergrund}

	\subsection{Gebundener Stickstoff}
	\label{ssec:Gebundener Stickstoff}

	In der Natur geschieht eine fortwährender Austausch von Stickstoff zwischen Lebewesen, Boden und Atmosphäre.
	Ein Großteil des vorhandenen Stickstoffs liegt in gebundener Form im Erdboden vor und ist ein unentbehrlicher Nährstoff für Pflanzen und Lebewesen.
	Er wird von Pflanzen beim Wachstum aus der Erde aufgenommen und beim Absterben wieder freigesetzt.
    Zur Steigerung der Ernte ist eine Anreicherung des Bodens mit Nitrat durch der Einsatz von Düngemitteln daher gängige Praxis in der Landwirtschaft.\cite{Umweltbundesamt2017}
    Dies kann zu Problemen führen, wenn es zu einer Übersättigung des Bodens mit Stickstoff kommt.
    Durch Ausschwemmen des in Form von Nitrat ($NO_3^-$) und Ammonium ($NH_4^+$) gebundenen Stickstoffs kann dieser ins Grundwasser gelangen und eine Gefahr für die Umwelt darstellen.
    Sowohl für den Erfolg der Landwirtschaft als auch für den Schutz der Umwelt ist daher eine zuverlässige und effiziente Ermittlung des Nitratgehalts im Boden von entscheidender Bedeutung.
    Der Messung des Stickstoffs liegen folgende chemische Zusammenhänge zu Grunde:


    Die Stoffmengenkonzentration von Stickstoff $c^\m{(N)}$ lässt sich berechnen durch
    	\[
			c^\m{(N)} \define \frac{n^\m{(N)}}{V}
		\]
		wobei $V$ das Volumen der Lösung und $n^\m{(N)}$ die enthaltenen Stoffmenge von Stickstoff ist.

    Für eine gegebene Stoffmengenkonzentration $c_0$ und Stoffmenge $n_o$ einer Probe, definieren wir den Stoffmengenanteil $y^\m{(N)}$ von Stickstoff als,
        \[
			y^\m{(N)} \define \frac{c^\m{(N)}}{c_0} = \frac{n^\m{(N)}}{n_0}
		\]

	% subsection Gebundener Stickstoff

	\subsection{Nahinfrarotspektroskopie}
	\label{ssec:Nahinfrarotspek}

		Bei der Nahinfrarotspektroskopie kommen elektromagnetische Wellen im Bereich zwischen 120 THz und 400 THz bzw. 2.500 nm und 750 nm zum Einsatz.\cite{Agelet2010}
		Das Messverfahren nutzt die Tatsache, dass bei der Bestrahlung einer Probe Teile des Lichts reflektiert, hindurch gelassen oder absorbiert werden.
		Von besonderem Interesse ist hierbei die Reflexion des Lichts, welche sich in die zwei Komponenten \glqq Spiegelreflexion\grqq{} und \glqq diffuse Reflexion\grqq{} unterscheiden lässt.
		Aus den Teilen des Lichts, welche diffus reflektiert werden, können - aufgrund der größeren Eindringtiefe - Informationen über die Beschaffenheit der Probe gewonnen werden.\cite{Agelet2010}
		Für eine Wellenlänge $\lambda$ ist das relative Reflexionsvermögen $\delta(\lambda)$ definiert als:
		\[
			\delta(\lambda) \define \frac{P_\m{r}(\lambda)}{P_s}
		\]

		wobei $P_\m{r}(\lambda)$ die Reflexion der Wellenlänge $\lambda$ einer Probe, und $P_0$ die Reflexion eines Materials mit einem Reflexionsanteil nahe $100\%$ ist.

		Weiterhin lässt sich unter Anwendung des Beer'schen Gesetzes ein approximativer Zusammenhang zwischen dem relativen Reflexionsvermögen und der Stoffmengenkonzentration $c^\m{(N)}$ des Stickstoffs herstellen.
		In eine Probe mit $n$ verschiedenen Stoffen sei $c_i$ die Stoffmengenkonzentration des $i$ten in der Probe enthaltenen Stoffes.
		Dann sei $\varepsilon_i(\lambda)$ ein Koeffizient mit $i\in\SN,i\leq n$ sodass
		\[
			-\log \delta(\lambda) = -\log \frac{P_\m{r}(\lambda)}{P_s} = \sum_{i=1}^{n} \varepsilon_i(\lambda) c_i
		\]


	% subsection Nahinfrarotspek

% section Hintergrund
