\section{Background}
\label{sec:background}
	
	\subsection{Gebundener Stickstoff}
	\label{ssec:Gebundener Stickstoff}
	
	Der Stickstoffkreislauf steht in engem Zusammenhang mit dem Gehalt des organischen Kohlenstoffes im Boden und hat. 
    Als Bestandteil von Aminosären in Proteinen und vielen weiteren zum Leben wichtige chemische Verbindungen ist der Stickstoffgehalt des Bodens ein wichtiger Indikator für Fruchtbarkeit des Bodens ... essentieller Bestandteil für den Anbau von Pflanzen.\\
		
	
	% subsection Gebundener Stickstoff

	\subsection{Nahinfrarotspektroskopie}
	\label{ssec:nirs}
	
		Bei der Nahinfrarot Spektroskopie kommen elektromagnische Wellen im Bereiche zwischen 120 und 400 THz zum Einsatz. \cite{Agelet2010}
		Das Messverfahren nutzt die Tatsache, dass bei der Bestrahlung einer Probe mit Licht Teile des Lichts reflektiert, hindurchgelassen oder absorbiert werden.
		Die Anwendung von Beers Law beschreibt die Korrelation zwischen der Konzentration einer Probe und deren Absorbtion bestimmter Wellenlänge.
		Obwohl das absorbierte Licht nicht direkt gemessen werden kann, können die diffuse Reflektion und Durchlässigkeit wie folgt in Zusammenhang gebracht werden.
		
		
		Summarizing, NIR measurements for analytical purposes
can be carried out in two modes: transmission or diffuse
reflection. Diffuse reflection mode allows working with thicker
and denser samples without inducing as much heating as trans-
mission. While sample path length is pre-determined and must
be kept constant for transmittance measurements, the minimum
sample required in reflectance mode is highly dependent on
the wavelength range used in the analysis and sample charac-
teristics such as density or packing, particle size, and material
		
		 
		The reflectance 
		\[
			\func{\varrho}{(0,\infty)}{(0,\infty)},\qquad \varrho(\lambda) \define \frac{P_\m{r}(\lambda)}{P_0}
		\]
		of a 

	% subsection nirs

% section fundamentals