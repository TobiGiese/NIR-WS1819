\section{Hintergrund}
\label{sec:Hintergrund}
	
	\subsection{Gebundener Stickstoff}
	\label{ssec:Gebundener Stickstoff}
	
	In der Natur geschieht eine fortwährender Austausch von Stickstoff zwischen Lebewesen, Boden und Atmosphäre. 
	Ein Großteil des vorhandenen Stickstoffs liegt in gebundener Form im Erdboden vor und ist ein unentbehrlicher Nährstoff für Pflanzen und Lebewesen. \textbf{Quelle}
	Er wird von Pflanzen beim Wachstum aus der Erde aufgenommen und beim Absterben wieder freigesetzt.
    Zur Steigerung der Ernte ist eine Anreicherung der Bodens mit Nitrat durch der Einsatz von Düngemitteln daher gängige Praxis in der Landwirtschaft.(\textbf{Quelle...}
    Dies kann dann zu Problemen führen, wenn zu einer Übersättigung des Bodens mit Stickstoff kommt.
    Durch Ausschwemmen des in Form von Nitrat ($NO_3^-$) und Ammonium ($NH_4^+$) gebundenen Stickstoffes kann dieses ins Grundwasser gelangen, was in höherer Konzentration eine Gefahr für die Umwelt darstellt.
    Für den Erfolg der Landwirtschaft als auch für den Schutz der Umwelt ist daher eine zuverlässige und effiziente Ermittlung des Nitratgehalts im Bodens von entscheidender Bedeutung.
    Die Stoffmengenkonzentration von Stickstoff ($N$) lässt sich berechnen durch,
    	\[
			c_{(N)} \define \frac{n_{(N)}}{V_P}
		\]
		,wobei $V$ das Volumen der Lösung und $N_{(N)}$ die enthaltenen Stoffmenge von Stickstoff ist.
		
    Für eine gegebene Stoffmengenkonzentration $c_0$ und Stoffmenge $n_o$ der gesamten Probe, definieren wir dann den Stoffmengenanteil $X_{(N)}$ von Stickstoff als,
        \[
			X_{(N)} \define \frac{c_{(N)}}{c_0} = \frac{n_{(N)}}{n_0}
		\]
   
	% subsection Gebundener Stickstoff

	\subsection{Nahinfrarotspektroskopie}
	\label{ssec:nirs}
	
		Bei der Nahinfrarot Spektroskopie kommen elektromagnetische Wellen im Bereich zwischen 120THz und 400 THz zum Einsatz. \cite{Agelet2010}
		Das Messverfahren nutzt die Tatsache, dass bei der Bestrahlung einer Probe mit Licht Teile des Lichts reflektiert, hindurch gelassen oder absorbiert werden.
		Von besonderem Interesse ist hierbei die Reflexion des Lichts.
		Dieses lässt in die zwei Komponenten Spiegelreflexion und diffuse Reflexion unterscheiden.
		Da die Teile des Lichts welche diffus reflektiert werden tiefer in die Proben eindringen, können aus diesen Informationen über dessen Zusammensetzung gewonnen werden. 
		Für eine Wellenlänge $\lambda$ ist das relative Reflexionsvermögen $\delta(\lambda)$ definiert als:
		\[
			\func{\delta}{(0,\infty)}{(0,\infty)},\qquad \delta(\lambda) \define \frac{P_\m{r}(\lambda)}{P_s}
		\]
		
		wobei $P_\m{r}(\lambda)$ die Reflexion einer Probe und $P_0$ die Reflexion eines Material mit einer Reflexion nahe $100\%$ ist.
		
		Weiterhin lässt sich ein Zusammenhang zwischen dem relativen Reflexionsvermögen und der Absorption herstellen.
		Unter Anwendung des Beer'schen Gesetzes kann das Reflexionsvermögen mit der Stoffmenge $c_{(N)}$ des Stickstoffs in Verbindung gebracht werden.
		In eine Probe mit $n$ verschiedenen Stoffen sei $c_i$ die Stoffmengenkonzentration des $i$ten Stoffes. Dann sei $\varepsilon_i(\lambda)$ ein Koeffizient mit $i\in\SN,i\leq n$ so, dass
		
		\[
			-\log \delta(\lambda) = -\log \frac{P_\m{r}(\lambda)}{P_s} = \sum_{i=1}^{n} \varepsilon_i(\lambda) c_i
		\]


	% subsection nirs

% section Hintergrund