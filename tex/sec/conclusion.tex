\section{Fazit}
\label{sec:Fazit}

    In der vorliegenden Arbeit wurde mit Hilfe von Mallow's $C_\m{p}$-Kriterium eine Prognosefunktion für Stoffmengenanteile von Stickstoff durch die Messung von Reflexionswerte verschiedener Wellenlängen im Nahinfrarotbereich hergeleitet.
    Hierbei wurde zunächst eine Vorauswahl relevanter Wellenlängen nach verschiedenen Kriterien getroffen und jenes Kriterium ausgewählt welches, bei weiterer Featureselektion zu dem kleinsten $SPSE$ in Kombination mit einem minimalen $C_{p}$ des Modells führte.
    Es zeigte sich, dass die Variabilität der differenzierten Reflexionswerte ein geeignetes Kriterium der Selektion darstellt.
    
    Um die tatsächliche Eignung des gewählten Modells zu überprüfen wurde zunächst das Bestimmtheitsmaß $R^2$ untersucht.
    Zur Evaluation der Schätzung des erwarteten Prognosefehlers $SPSE$ wurde darüber hinaus Pseudobeobachtungen durch das gewählte Modell generiert.
    Die Schätzwerte des erwarteten Prognosefehlers verdeutlichen den starken Einfluss des Stichprobenumfangs auf dessen Genauigkeit.
    Die Ergebnisse der Untersuchungen zeigen außerdem die Notwendigkeit einer sorgfältigen Auswahl der Einflussgrößen für das Maximalmodell.
    In diesem Zusammenhang könnten in Zukunft auch andere Vorgehensweisen für die Selektion der Einflussgrößen getestet werden.
    Denkbar wäre beispielsweise eine Formulierung des Problems als Minimierungsproblem.\cite{Menzel2016}
    
    Insgesamt lässt sich zusammenfassen, dass mit dem gewählten Vorgehen eine geeignete Prognosefunktion bzw. Kalibrierung gefunden wurde.
    
% section Fazit