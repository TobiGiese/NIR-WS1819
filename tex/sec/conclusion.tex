\section{Fazit}
\label{sec:Fazit}

    In der vorliegenden Arbeit wurde mit Hilfe von Mallow's $C_\m{p}$ eine Prognosefunktion für Stoffmengenanteile von Stickstoff durch die Messung von Reflexionswerte verschiedener Wellenlängen im Nahinfrarotbereich hergeleitet.
    Hierbei wurde zunächst eine Vorauswahl relevanter Wellenlängen nach verschiedenen Kriterien getroffen und jenes ausgewählt welches, bei weiterer Featureselection zu dem kleinsten $SPSE$ in Kombination mit einem minimalen $C_{p}$ des Modells führte.
    Es zeigte sich, dass die Variabilität der differenzierten Reflexionswerte ein geeignetes Kriterium der Selektion darstellt.
    
    Um die tatsächliche Eignung des gewählten Modells zu überprüfen wurde zunächst das korrigierte Bestimmtheitsmaß $R^2$ untersucht.
    Zudem wurden mehrere Simulationen durchlaufen bei denen zunächst Pseudobeobachtungen durch das gewählte Modell generiert wurden um die Schätzung des erwarteten Prognosefehlers $SPSE$ zu evaluieren.
    Es wurde deutlich, das der Umfang der Stichprobe einen starken Einfluss auf Genauigkeit des Schätzwertes hat.
    Dies verdeutlich außerdem die Notwendigkeit einer sorgfältigen Auswahl der Einflussgrößen für das Maximalmodell.
    
    Insgesamt lässt sich zusammenfassen, dass ....
    
	

% section Fazit