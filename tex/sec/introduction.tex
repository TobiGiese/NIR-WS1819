\section{Einleitung}
\label{sec:Einleitung}

    Die Zusammensetzung des Boden ist ein wesentlicher Faktor für einen erfolgreichen und nachhaltigen Anbau von Pflanzen in der Landwirtschaft.
    Durch die Messung des im Boden gebundenen Stickstoffes und organischen Kohlestoffes (SOC) können nicht nur Fragen über die Fruchtbarkeit des Bodens, sondern auch Erkenntnisse über den Effekt der Bodennutzung auf diese Werte beantwortet werden.
    Problematisch stellen sich hierbei allerdings die aufwändigen und fehleranfälligen Messmethoden dar.
    Aus diesem Grund sind Messverfahren notwendig, welche die zuverlässige und effiziente Ermittlung der genannten Parameter zulassen.
    Ein viel diskutiertes Messverfahren ist die sogenannten Nahinfrarot Spektroskopie.
    
    
    , was nach einer effizienteren und zuverlässigeren Methoden zur Ermittlung der genannten Parameter verlangt.
    Deren Messung ist allerdings aufwändig und fehleranfällig
    Umso wichtiger ist deren korrekte und effiziente Messung.

    Die Zusammensetzung von Bodenproben 
	
	Der Stickstoffkreislauf steht in engem Zusammenhang mit dem Gehalt des organischen Kohlenstoffes im Boden und hat. 
	Als Bestandteil von Aminosären in Proteinen und vielen weiteren zum Leben wichtige chemische Verbindungen ist der Stickstoffgehalt des Bodens ein wichtiger Indikator für Fruchtbarkeit des Bodens ... essentieller Bestandteil für den Anbau von Pflanzen.\\
	
	Herausforderung der Messung des Stickstoff- bzw. Bodenkohlenstoffgehalts\\
	Die Messung des Stickstoffgehalts des Bodens erfordert ein chemisches Messverfahren
    \\
    
    Lösung\\
    Einsatz von Nahinfrarot Spektroskopie zur 
	
% section Einleitung