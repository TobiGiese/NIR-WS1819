\section{Einleitung}
\label{sec:Einleitung}

    Die Zusammensetzung des Boden ist ein wesentlicher Faktor für einen ertragreichen und nachhaltigen Anbau von Pflanzen in der Landwirtschaft.
    Wichtige Parameter hierfür sind die Anteile des im Boden gebundenen Stickstoffs (N) und organische Kohlenstoffs (SOC), durch deren Messung Erkenntnisse über die Fruchtbarkeit des Bodens und die Auswirkungen der Bodennutzung gewonnen werden können.\cite{Poeplau2013}
    Die Messung der genannten Parameter ist durch sogenannte Fraktionierung von Bodenprobe möglich.
    Etablierte Messverfahren sind jedoch kostenintensiv, nicht vollständig standardisiert, und weisen eine schlechte Reproduzierbarkeit in verschiedenen Laboren auf.\cite{Poeplau2013}
    Aus diesem Grund sind Messverfahren notwendig, welche eine zuverlässige und effiziente Ermittlung der Menge des organischen Kohlenstoffs und gebundenen Stickstoffs im Bodens zulassen.
    Ein seit den sechziger Jahren vielfach eingesetztes Messverfahren ist die sogenannten Nahinfrarotspektroskopie. \cite{Agelet2010}
    Dieses erlaubt die Schätzung der aufwendig bestimmbaren Parameter mit Hilfe der leicht messbaren Nahinfrarotspektren.
    Die Bestimmung und Validierung eines hierfür notwendigen Modells für den Stickstoffgehalt in der Bodenprobe sind Gegenstand dieser Arbeit.

% section Einleitung