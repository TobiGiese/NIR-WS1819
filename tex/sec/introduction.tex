\section{Einleitung}
\label{sec:Einleitung}
    Die Zusammensetzung des Boden ist ein wesentlicher Faktor für einen ertragreichen und nachhaltigen Anbau von Pflanzen in der Landwirtschaft.
    Wichtige Parameter hierfür sind die Anteile des im Boden gebundenen Stickstoffs (N) und organischen Kohlenstoffs (SOC), durch deren Messung Erkenntnisse über die Fruchtbarkeit des Bodens und die Auswirkungen der Bodennutzung gewonnen werden können.\cite{Poeplau2013}
    Die Messung der genannten Parameter ist durch sogenannte Fraktionierung von Bodenproben möglich.
    Etablierte Messverfahren sind jedoch kostenintensiv, nicht vollständig standardisiert, und weisen eine schlechte Reproduzierbarkeit in verschiedenen Laboren auf.\cite{Poeplau2013}
    Aus diesem Grund sind Verfahren notwendig, welche eine zuverlässige und effiziente Ermittlung der Menge des organischen Kohlenstoffs und gebundenen Stickstoffs im Boden zulassen.
    Ein seit den sechziger Jahren vielfach eingesetztes Messverfahren, ist die sogenannten Nahinfrarotspektroskopie.\cite{Agelet2010}
    Dieses erlaubt die Schätzung der nur aufwendig bestimmbaren Einflussparametern auf den Anteil von N (bzw. SOC) in Bodenproben mit Hilfe der leicht messbaren Nahinfrarotspektren.
    Die Bestimmung und Validierung eines hierfür notwendigen statistischen Modells für den Stickstoffgehalt in der Bodenprobe, ist Gegenstand dieser Arbeit.
    Darüber hinaus widmet sich diese Arbeit einer statistischen Simulationsaufgabe.
    Dabei soll untersucht werden, inwiefern sich der über Mallow's $C_p$-Kriterium geschätzte erwartete Prognosefehler (siehe Kapitel \ref{sec:Methodik}) in Abhängigkeit der Stichprobengröße verändert.

% section Einleitung