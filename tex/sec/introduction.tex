\section{Einleitung}
\label{sec:Einleitung}

    Die Zusammensetzung des Boden ist ein wesentlicher Faktor für einen erfolgreichen und nachhaltigen Anbau von Pflanzen in der Landwirtschaft.
    Wichtige Parameter hierfür sind der organische Kohlenstoff des Bodens sowie der im Boden gebundene Stickstoff. Durch die Messung dieser Werte können nicht nur Fragen über die Fruchtbarkeit des Bodens, sondern auch Erkenntnisse über den Effekt der Bodennutzung auf diese Werte beantwortet werden.\cite{Poeplau2013}
    Etablierte Messverfahren lassen sich in physische und chemische Messverfahren unterteilen. Diese sind jedoch kostenintensiv und   nicht vollständig standardisiert, sodass ihre Reproduzierbarkeit in verschiedenen Laboren nicht garantiert ist.
    Aus diesem Grund sind Messverfahren notwendig, welche die zuverlässige und effiziente Ermittlung organischen Kohlenstoffs und gebundenen Stickstoffs des Bodens zulassen.
    Ein seit den sechziger Jahren vielfach eingesetztes Messverfahren ist die sogenannten Nahinfrarot Spektroskopie. \cite{Agelet2010}
    Diese erlaubt die Messung der schwierig zu messenden Parameter N und SOC mit Hilfe der leicht messbaren Nahinfrarotspektren.

	
% section Einleitung