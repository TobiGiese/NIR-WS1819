\section{Einleitung}
\label{sec:Einleitung}

    Die Zusammensetzung des Boden ist ein wesentlicher Faktor für einen ertragreichen und nachhaltigen Anbau von Pflanzen in der Landwirtschaft.
    Wichtige Parameter hierfür sind der organische Kohlenstoff des Bodens sowie der im Boden gebundene Stickstoff, durch deren Messung Fragen über die Fruchtbarkeit des Bodens, sowie Erkenntnisse über den Effekt der Bodennutzung beantwortet werden können.\cite{Poeplau2013}
    Die Messung der genannten Parameter ist durch sogenannte Fraktionierung der Bodenprobe möglich.
    Etablierte Messverfahren sind jedoch kostenintensiv, nicht vollständig standardisiert, und weisen eine schlechte Reproduzierbarkeit in verschiedenen Laboren auf.\cite{Poeplau2013}
    Aus diesem Grund sind Messverfahren notwendig, welche die zuverlässige und effiziente Ermittlung organischen Kohlenstoffs und gebundenen Stickstoffs des Bodens zulassen.
    Ein seit den sechziger Jahren vielfach eingesetztes Messverfahren ist die sogenannten Nahinfrarot Spektroskopie. \cite{Agelet2010}
    Diese erlaubt die Schätzung der aufwendig bestimmbaren Parameter N und SOC mit Hilfe der leicht messbaren Nahinfrarotspektren.

	
% section Einleitung